\section{Conclusion}
In this term paper, we have introduced the core topics of Bayesian statistics and provided, where possible, a comparison to their frequentist counterpart. We have then developed the concept of hierarchical data and modelling. We have shown the general structure of hierarchical  models and then carried over this theory into section \ref**Markus where we have showcased the properties of Bayesian modelling in a monte carlo simulation study. In the simulation study, we demonstrated the effect of various prior distributions on the estimation of hierarchical models. We have shown that wrong prior distribution pulls the posterior in its direction. However, the prior also guides the estimation of parameters. We have also found that a uniform prior does increase the variability in the model estimates. 
We further demonstrate the applicability of Bayesian estimation, as compared to a frequentist maximum likelihood approach, on hierarchical models through a real world data example. We have used a weak prior distribution to estimate a linear multilevel model on the data.  The results show that Bayesian estimation propagates the uncertainty in the hyperparameters throughout all levels of the model and therefore provides more appropriate estimates of uncertainty \cite{browne2006comparison}. The difference in between- and within school variance however, is not that large. It remains to be investigated in further studies, once the number of varying slopes/intercepts increase or in the case where we have a non-nested data structure,  how the differences would change. We conclude that it is a good strategy to use weak prior distributions, even when one is uninformed about the actual parameter values. 
