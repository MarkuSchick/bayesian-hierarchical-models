\section{Hierarchical Models}
In the following subsections we will introduce the concept of hierarchical data
and the models designed to work with that data. We present an analytical
derivation for a simple but general case and then illustrate the most common
model class, \emph{linear hierarchical models}.

\subsection{Hierarchical Data}
Hierarchical data is present when there is a natural way to split observations
into clusters. For example, we might observe data on children for many different
schools. The data could also include schools for different states. So we would
observe children in schools in states. This would
constitute the case of \emph{nested} groups, which gives meaning to the term
hierarchical. There are however many cases which do not feature nested data
or an interpretable hierarchical structure but do belong to the type of structured
data that can be analyzed using hierarchical models. A prominent example are
meta-analysis studies, which can be modeled hierarchically but are non-nested
since units might be overlapping. When a clear interpretation of the hierarchy is missing
one often encounters the description \emph{multi-leveled data} and \emph{multi-level model}.

\subsection{Solving for the posterior analytically}
As in the previous section, we will first present the analytical derivation of
the posterior in a simple normal hierarchical model. We continue using the
example of modeling observed test scores of children in different schools.

Let us assume that we observe test scores $y_i$ for $i = 1,\mydots,n$ children
in $j=1,\mydots,J$ different schools. For convenience let us write $j[i]$ for child
$i$'s school. We assume that for all children in school $j$, the outcome $y_i$
follows a common data generating process governed by some parameter $\theta_j$.
To make this model hierarchical we assume that these $\theta_1,\mydots,\theta_J$
also follow a common data generating process governed by some hyperparameter $\phi$.
This gives
\begin{align}
  p(y_i \mid \theta_{j[i]}, \phi) = p(y_i \mid \theta_{j[i]})
\end{align}
for our model of test scores given all parameters. Note that since $\phi$ only
affects $y_i$ through $\theta_{j[i]}$ we can drop it from the conditioning set.
Further we model the common structure of the $\theta_j$'s through $p(\theta_j \mid \phi)$.
To analyze this problem using Bayesian techniques we must at last assign a prior
to $\phi$, that is, we define $p(\phi)$.

The objects of our interest are given by the joint posterior $p(\theta_j, \phi \mid y)$
and its two marginal posteriors. In order to derive these analytically we must
make strong distributional assumptions.

\subsection{Hierarchical Linear Models}

\subsubsection{Varying Slopes Model With One Predictor In Each Level}
We assume that units $i = 1,\mydots,n$ can be divided into $J$ distinct groups.
We start with a very simple model assuming that intercept is fixed for all
groups, that is
\begin{align}
  y = \alpha + \beta_j x + \epsilon \,,
\end{align}
with $\epsilon$ following a mean zero normal distribution with variance
$\sigma_{\epsilon}^2$.
To incorporate the idea that the groups follow a common structure we also
assume
\begin{align}
  \beta_j &= \gamma_0 + \gamma_1 u_j + \eta \,,
\end{align}
for $j = 1,\mydots,J$, with $\eta$ mean zero normal with variance $\sigma_\eta^2$.

Since $\gamma_0$ and $\gamma_1$ do not vary by group they are sometimes referred
to as \emph{fixed effects}. Similary as $\eta$ is drawn randomly for each group
it is sometimes called a \emph{random effect}. Put together this shows the close
resemblance of the hierarchical linear model to classical mixed effects models
\textcolor{red}{(some reference here would be nice!)}

Following the notation of \textcolor{red}{Gelman and Hill (2007)} we describe the
model equation of a single individual $i$ by
\begin{align}
  y_i = \alpha + \beta_{j[i]} x_i + \epsilon_i \,,
\end{align}
where $j[i]$ denotes the group to which individual $i$ belongs.

The model defined by the assumptions and equations above \textcolor{red}{(where and what)} can of course be made
arbitrarily complex. For example we could add higher order polynomial terms or
more predictors, as in regular linear regression modeling. Further, the
normality assumption of the errors could be relaxed and most importantly, why
stop at two levels? Naturally we could model the group-level coefficients using
a third level. For the sake of a simpler explaination however, we will stick to
the presented model.

%\subsubsection{Varying Intercept and Slope Model with One Predictor in Each Level}
%We assume that units $i = 1,\mydots,n$ can be divided into $J$ distinct groups.
%In each group $j$ we model our outcome variable $y$ as a linear function in $x$,
%that is
%\begin{align}
  %y = \alpha_j + \beta_j x + \epsilon \,,
%\end{align}
%with $\epsilon$ following a mean zero normal distribution with variance
%$\sigma_{\epsilon}^2$.
%To incorporate the idea that the groups follow a common structure we also
%assume
%\begin{align}
  %\alpha_j &= \gamma_0^{\alpha} + \gamma_1^{\alpha} u_j + \eta_\alpha \\
  %\beta_j &= \gamma_0^{\beta} + \gamma_1^{\beta} u_j + \eta_\beta \,,
%\end{align}
%for $j = 1,\mydots,J$, with
%\begin{align}
  %\sqmat{\eta_\alpha \\ \eta_\beta} \sim \normal{\sqmat{\sigma_\alpha^2 & \rho \sigma_\alpha \sigma_\beta \\ & \sigma_\beta^2}}
%\end{align}
%
%Following the notation of \textcolor{red}{Gelman and Hill (2007)} we describe the model equation of a single individual $i$ by
%\begin{align}
  %y_i = \alpha_{j[i]} + \beta_{j[i]} x_i + \epsilon_i \,,
%\end{align}
%where $j[i]$ denotes the group to which individual $i$ belongs.
%This particular model is known as the two-level varying intercept / varying slope model with one unit-level predictor (here $x_i$) and one group-level predictor
%(here $u_i$).
%The model defined by equations \textcolor{red}{(1) - (4)} can of
%course be made arbitrarily complex by adding higher order polynomial terms or
%more predictors, as in a regular linear regression models. Further the normality
%assumption of equation \textcolor{red}{(4)} is not mandotory and can be swapped
%with nearly any other distributional assumption. Also, why stop at two levels?
%We could naturally model the coefficients in equation \textcolor{red}{2 and 3}
%using a third level. All these extensions can in practice be necessary when
%modelling complex structures; however for the sake of simplicity and clarity
%we will stick to our clean model.
