\section{Application}

\begin{frame}{The Data}
	\Large{
	\begin{itemize}
		\item[] \emphcol{Description:} General Certificate of Secondary Education (GCSE) exam scores of 1,905 students from 73 schools in England on a science subject
		\item[] \emphcol{Variables of interest:} school identifier, student identifier, gender, total score on written paper and total score of course work.

	\end{itemize}
	}
\end{frame}


\begin{frame}{ML and Bayesian Approach: Comparison}
	\Large{
		\emphcol{The model:} Varying intercept and slope model with a single predictor
		\begin{align}
			y_{ij} &= \alpha_j + \beta_j x_{ij} +\epsilon_{ij},\\[0.5em]
			\alpha_j &= \mu_\alpha + u_j,\\[0.5em]
			\beta_j &= \mu_\beta + v_j,\\[0.5em]
			y_{ij} &= \mu_\alpha + \mu_\beta x_{ij} + u_j + v_j x_{ij} + \epsilon_{ij}
		\end{align}
	}
\end{frame}

\begin{frame}{ML and Bayesian Approach: Comparison}
	\Large{
		\emphcol{Maximum Likelihood (ML) Estimation}
		\begin{itemize}
		\item[] \emphcol{Package:} \textit{lmer}
		\item[] \emphcol{The \textit{lmer} Package:} combines of ML estimation of model parameters and empirical Bayes (EB) predictions of the varying intercepts and/or slopes resulting in the Best Linear Unbiased Predictions (BLUPs) of the model parameters.
		\item[] \emphcol{Why lmer??} allows for comparison between parameter estimates
		\end{itemize}
	}
\end{frame}

\begin{frame}{ML and Bayesian Approach: Comparison}
	\Large{
		\emphcol{Bayesian Estimation:}
		\begin{itemize}
			\item[] \emphcol{Package:} \textit{stan}
			\item[] \emphcol{Priors:} weekly informative normally distributed priors for hyperparameters

		\end{itemize}
	}
\end{frame}

\begin{frame}{ML and Bayesian Approach: Results}
	\Large{
			\begin{columns}

				\begin{column}{0.7\textwidth}
					\begin{table}
						\footnotesize
						\begin{tabular}{l*{2}{c}}
							\toprule \\[-1.0em]
							\multicolumn{3}{c}{Dependent variable: Course test score}\\ \\[-1.0em]
							&ML &Bayes\\ \midrule \\[-1.0em]
							\emph{A. Random effects} \\
							Intercept & -- & --\\
							& (10.146) & (10.249)\\ \\[-1.0em]
							Female & -- & --\\
							& (6.924) & (7.099)\\ \\[-1.0em]
							\\ \\[-1.0em]\emph{B. Fixed effects} \\
							Intercept & 69.425 & 69.413\\
							& (1.352) & (1.287)\\ \\[-1.0em]
							Female & 7.128 & 7.132\\
							& (1.131) & (1.165)\\ \\[-1.0em]
							\\ \\[-1.0em]\hline \\[-1.0em]
							\emph{N} \\
							\hspace{3mm}Students&1725&1725\\
							\hspace{3mm}Schools&73&73\\
							\bottomrule
						\end{tabular}
					\end{table}
				\end{column}

			\hspace{-20pt}
			\begin{column}{0.5\textwidth}
				\begin{itemize}
					\item[\emphcol{(i.)}] point estimates almost the same
					\item[\emphcol{(ii.)}] Bayes standard deviations for random effects may be higher because ML does not take into account group level variance
				\end{itemize}
			\end{column}

			\end{columns}


	}

\end{frame}

\begin{frame}{Bayesian approach - further analysis}
	\Large{
		\vfill
		\emphcol{Posterior distribution ranking:}
		\begin{figure}
			\centering
			\includegraphics<1>[height=7cm]{graphics/ranking}
		\end{figure}
		\vfill
	}
\end{frame}

\begin{frame}{Bayesian approach - further analysis}
	\Large{
		\vfill
		\emphcol{School specific regression lines and pooling:}
		\begin{figure}
			\centering
			\includegraphics<1>[height=7cm]{graphics/pooling}
		\end{figure}
		\vfill
	}
\end{frame}

\begin{frame}{Bayesian approach - further analysis}
	\Large{
		\vfill
		\emphcol{Making comparisons between individual schools:}
		\begin{figure}
			\centering
			\includegraphics<1>[height=7cm]{graphics/differences}
		\end{figure}
		\vfill
	}
\end{frame}

\begin{frame}{Bayesian approach - further analysis}
	\Large{
		\vfill
		\emphcol{Convergence:}
		\begin{figure}
			\centering
			\includegraphics<1>[height=5cm]{graphics/rhat}
		\end{figure}
		\vfill
	}
\end{frame}

\begin{frame}{Bayesian approach - further analysis}
	\Large{
		\vfill
		\emphcol{Convergence:}
		\begin{figure}
			\centering
			\includegraphics<1>[height=7cm]{graphics/ess}
		\end{figure}
		\vfill
	}
\end{frame}
