\section{Application}

In this section we discuss the applications of the hierarchical modelling approach in the form of a literature review. We then use real world data to give a practical illustration of the work we have developed in the previous sections of our paper and compare that to the frequentist approach to analyzing multilevel data with the aim of showcasing what differences/similarities there are between the two methods. We then depart form the comparative analysis to look deeper into the Bayesian approach and conduct robustness checks that vary certain aspects (e.g. priors) of it so that we can see how the estimation results change in response to that. The baseline here would be the estimation results from the comparison between Bayesian and frequentist approaches. We close the section off by a discussion of the limitations and challenges encountered in this section.


\subsection{Literature Review on the application of Hierarchical Models}

Bayesian Hierarchical models or multilevel models are a suitable approach to consider social contexts as well as individual respondents or subjects. It becomes attractive to consider hierarchical models in place of the common (or popularized) frequentist approach as soon as there is a need to relax the independence of residuals assumption as a result of similarities in the characteristics of a group of respondents or when the researcher seeks to disentangle variability at various levels of the data. These models have been used in various applications throughout fields in economic research. In education research, \cite{buric2020teacher} use these models to examine the relationship between teacher self-efficacy (TSE), instructional quality (i.e., classroom management, cognitive activation, and supportive climate) and student motivational beliefs (i.e., self efficacy and intrinsic motivation) by using responses from both teachers and students and implementing a sophisticated doubly latent multilevel structural equation modelling approach. The results reflect the necessity to disentangle variability at various levels of the data as the researchers find that, at class level, TSE was positively related to the three dimensions of instructional quality but not to students' motivational beliefs. They also find, as expected, that instructional quality was positively related to students’ motivational beliefs.

In family economics, \cite{lamnisos2019demographic} project the total fertility rate and life expectancy at birth probabilistically using Bayesian hierarchical models and United Nations population data for Greece from the period of 1950 to 2015. These are then converted to age-specific mortality rates and combined with a cohort component projection model. This yields probabilistic projections of total population by sex and age groups, total fertility rate (TFR), female and male life expectancies at birth and potential support ratio PSR (persons aged 20-64 per person 65+) by the year 2100. If the forecasts prove in future to be accurate, these models can provide a powerful tool for policy formulation. In agricultural research, \cite{ ramsey2019saying} develop two econometric models: a hierarchical Bayesian linear model and a hierarchical Bayesian Poisson model to predict exit rates across the towns and prefectures of Japan resulting from off-farm employment opportunities. Off-farm employment opportunities are thought to have an effect on farm exit rates, though evidence on the sign of this effect has been mixed. Examining this issue in the context of Japanese agriculture, the researchers find that farm exits are related to off-farm income as a share of household income, and more specifically to the nature of off-farm work.  

In development economics, \cite{ meager2019understanding} jointly estimates the average effect and the heterogeneity in effects across seven studies using Bayesian hierarchical models to answer questions about external validity that impede consensus on the results from randomized evaluations of microcredit. The researcher finds reasonable external validity: true heterogeneity in effects is moderate, and approximately 60 percent of observed heterogeneity is sampling variation. These paper has the potential to revolutionize the field of development economics as the researcher provides a method to establish external validity using multiple studies form different countries. In health research, \cite{ rashid2019socio} uses a more advanced application of Bayesian Hierarchical Models in health research. The authors aim to identify the spatial distribution of the three types of misconception factors of HIV transmission (i.e. transmitted by mosquito bite, supernatural means and sharing food with HIV positive person). This study also provides the core socio-economic factors to stop the misconception about HIV/ Aids transmission and helped in reducing its epidemic in Pakistan. Spatial and Non-Spatial Bayesian Hierarchical model were applied to the data and results from them revealed that the Conditional Autoregressive Bayesian Hierarchical Models (Spatial Model) were more appropriate. The results showed that Conditional Autoregressive Bayesian Hierarchical models at level 2 are best fit to the data.

It is evident that Bayesian Hierarchical Models can be useful in the area of microeconomic research. There also have been applications to the Macroeconomics field. It is evident that Bayesian Hierarchical Models can be useful in the area of microeconomic research. There also have been applications to the Macroeconomic research field. \cite{ koop2010bayesian} notes that bayesian methods have become increasingly popular as a way of overcoming over-parameterization problems. In this paper, the authors discuss vector autoregressive multivariate time
series models (VARs), factor augmented VARs and time-varying parameter extensions and show how Bayesian inference proceeds. 

We now demonstrate below, an aplication to real world data of the ------ model and compare the results to its frequestist model and estimation counterpart.

\subsection{Frequentist and Bayesian Approaches in Practice: Application to Education Data}

A common feature of data structures in education is that units of analysis (e.g., students) are nested in higher organizational clusters (e.g. schools). This kind of structure induces dependence among the responses observed for units within the same cluster. Students in the same school tend to be more alike in their academic and attitudinal characteristics than students chosen at random from the population at large. Multilevel models are designed to model such within-cluster dependence. As mentioned earlier, one advantage of multilevel models is that it allows us to disentangle variability between levels and in our data example,  multilevel models recognize the existence of data clustering (at two or more levels) by allowing for residual components at each level in the hierarchy. For example, a two-level model that allows for grouping of student outcomes within schools would include residuals at both the student and school level. The residual variance is thus partitioned into a between-school component (the variance of the school-level residuals) and a within-school component (the variance of the student-level residuals). 


We will be analyzing the Gcsemv dataset from \cite{rasbash2000user}. The data include the General Certificate of Secondary Education (GCSE) exam scores of 1,905 students from 73 schools in England on a science subject. The Gcsemv dataset consists of the following 5 variables:
\begin{itemize}
	\item \textit{school}: school identifier
	\item \textit{student}: student identifier
	\item \textit{gender}: gender of a student (M: Male, F: Female)
	\item \textit{written}: total score on written paper
	\item \textit{course}: total score on coursework paper
\end{itemize}

\subsubsection*{Visualization of the data}



\subsection{Looking Deeper into the Baysian Approach: Robustness Checks}

\subsection{Challenges and Limitations}






